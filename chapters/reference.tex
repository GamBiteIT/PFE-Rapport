\chapter*{Références} 
\label{chap:Références} 

[1] Bureau Business France d’ALGER , "Point sur l’agriculture en 2020 - Algérie" , Presse
  République Française Liberté Égalité Fraternité , 10 Janvier 2021 (\url{https://www.businessfrance.
  fr/algerie-point-sur-l-agriculture-en-2020}).\newline
  
  [2]\newline

[3] Module 4, Production de transplants et de légumes en serres - Chapitre 7, « Cultures
  en serre », manuscrit du Guide de gestion globale de la ferme maraîchère biologique et
  diversifiée, rédigé par Anne Weill et Jean Duval , page 4.\newline
  
[4] Article sur Nutrition de la tomate "Principes agronomiques de la tomate" , Knowl-
  edge grows yara france,paraghraph 2.\newline (\url{https://www.yara.frfertilisation/solutions-pour-cutomate/principes-agronomiques-tomates/}).\newline

[5] Article « La culture de tomates de A à Z » publie par Antoine de France Serres. (\url{https://www.france-serres.com/blog/culture/culture-de-tomates})  \newline

[6] Alexander S. Gillis, (https://www.techtarget.com/contributor/Alexander-S-Gillis) Technical Writer and Editor, “What Is IoT (Internet of Things) and How Does It Work?” IoT Agenda, TechTarget,11 Feb 2020. (\url{https://www.techtarget.com/iotagenda/definition/Internet-of-Things-IoT})\newline

[7] Article written by shivalibhadaniya and translated by Acervo Lima,”Caractéristiques de l’Internet des objets”,StackLima,juillet 5, 2022\newline
(\url{https://stacklima.com/caracteristiques-de-l-internet-des-objets/}).