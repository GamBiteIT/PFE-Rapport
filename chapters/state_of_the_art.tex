\chapter{Information general sur les serres } \label{chap:Information general sur les serres}

\section*{Introduction}
Ce chapitre fournit des informations de base sur la gestion des cultures en serre.
En savoir plus sur la production de tomates. 
Le succès des cultures dépend d'une bonne régulation de l'humidité,
 températur et bonne irrigation . le problème de la maladie 
 Souvent dû à un mauvais contrôle de l'humidité et de la ventilation en fonction des conditions météorologiques 
 que fait-il. Même les serriculteurs biologiques expérimentés peuvent 
 La difficulté de faire pousser des cultures avec succès.
 \\
\section*{Tomates de serre }
La culture des tomates en serre est un moyen de production de tomates efficace et peu coûteux en Algérie. La culture des tomates en serre offre aux agriculteurs un moyen sûr et pratique de produire des tomates de qualité supérieure à des coûts et à des quantités élevés.
Elle repose sur les principes suivants :
\section*{Caracteristique}
\section{température}
La température est le facteur le plus déterminant dans la production de la tomate. Celle-ci réagit énormément aux variations thermiques.
\\
Un écart de température d'un ou deux degrés Celsius entre le jour et la nuit est favorable à la production de fruits. Pour la production des plants et avant l'apparition des premières fleurs, il est préférable toutefois de maintenir la température plus constante à environ 20 °C. En été, l'aération doit être suffisante pour que la température ne dépasse pas 28 °C. Idéalement, il faudrait qu'elle ne dépasse pas 25 °C, car au dessus de cette température la tomate ne fait plus aucun gain. [2]
\\
\section{Humidité}
L'humidité relative optimale des cultures sous serre varie de 60 à 80 \%. Dans le cas des cultures hydroponiques, l'humidité relative nocturne et diurne se monte en général respectivement à 75 \% et 85 \%. [3] 
\section{Eclairage}
Les tomates sont sensibles aux conditions de faible luminosité. Elles exigent un minimum de 6 heures d'ensoleillement direct pour fleurir. Toutefois, en cas de trop grande intensité du rayonnement solaire, des fentes, des brûlures solaires et une coloration inégale peuvent apparaître au stade de maturité. Il est donc essentiel, dans le cas des cultures sous serre, de s'assurer que les fruits disposent de suffisamment d'ombre. La longueur du jour n'influence pas la production de tomates. Les cultures sous serre sont par conséquent répandues sous un large éventail de latitudes.[4]