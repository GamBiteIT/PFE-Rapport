\chapter*{Géneral Introduction} 
\label{chap:introduction} 
\addcontentsline{toc}{chapter}{Géneral Introduction}


\section*{Domaine d'étude}

L’agriculture a largement contribué à l’économie de l’Algérie pendant des siècles. L’Algérie
est l’un des principaux exportateurs de produits agricoles. Ces dernières années, le
gouvernement algérien a pris des mesures pour stimuler le secteur agricole en investissant
dans la technologie, les infrastructures et la formation et il s’agit notamment du Plan
national de développement agricole (PNDA), qui est conçu pour stimuler l’efficacité et
la compétitivité. Dans son rapport [1], le bureau d’étude « Bureau Business France d’ALGER»  présise qu’ il a été décidé d’orienter les efforts vers une agriculture intelligente et solide face au changement climatique en tenant compte de l’environnement et de l’équilibre des écosystèmes sans négliger le gaspillage, grâce à une bonne gestion des excédents de production.


\section*{Problématique }
Les serres intelligentes(SI) et l’Internet des Objets (IoT) sont des technologies émergentes
qui ont transformé l’agriculture. SI sont des structures utilisées pour cultiver des plantes
sous un environnement contrôlé, offrant ainsi une solution efficace pour la production
de cultures dans des conditions météorologiques défavorables ou dans des régions où les
terres sont limitées.
Les serres intelligentes utilisent des technologies de pointe pour optimiser les conditions
de croissance des plantes. Ces technologies peuvent inclure des capteurs de température,
d’humidité et de lumière, des systèmes d’irrigation automatisés et des logiciels d’analyse
de données pour suivre les conditions de croissance des plantes.SI permettent également
une surveillance et un contrôle précis des conditions environnementales.
L’utilisation de l’IoT dans les SI offre une connectivité en réseau aux appareils électroniques, des capteurs..., afin de collecter une grande quantité de données, pour qu’elles puissent être analysées et interprétées de manière efficace en faveur d’augmenter la productivité.

\section*{Objectifs }
Pour bien répondre a notre problématique, nous nous sommes fixés les objectifs suivants : 
	1.	Etudier les exigences pour l’obtention d’une bonne productivité agricole, en qualité et en quantité. Pour bien cerner cet objectif, nous concentrons cette étude sur la production de la 
tomate industrielle, en raison de sa forte productivité et la sa métrise en Algérie  Cette phase permettra de ressortir les paramètres importants pour atteindre les objectifs définie par un agriculteur .
\\
	2.	Retrouver les capteurs nécessaires qui répondent au mieux à la collecte des informations .
\\
    3.	Etudier les différents algorithmes d’analyse de données pour retrouver le plus adéquat à notre cas particulier .
\\
    4.	Proposer une solution à notre Smart Farm système, éventuellement proposer un algorithme spécifique de classification .
\\
    5.	Proposer une architecture pour une SmartFarm, permettant une bonne exploitation d’une ferme agricole .
\\
    6.	Tester le système sur un cas particulier sur un type de production agricole, la tomate par exemple.  

\section*{Organisation de mémoire }
Notre mémoire est structuré comme suite : 
\\
\textbf{Chapitre 1: }
Présentation générale de l’agriculture sous Serres
\\
\textbf{Chapitre 2: }
Présentation générale de l’IOT
\\
\textbf{Chapitre 3: }
Présentation générale du Conception d’Analyse des données
\\
\textbf{Chapitre 4: }
Proposition d’un système de contrôle d’une ferme intelligente « Smart Greenhouse »  
\\
\textbf{Chapitre 5:}
Implémentation de : Smart Greenhouse
\\
\textbf{Chapitre 6:}
Conclusion génerale