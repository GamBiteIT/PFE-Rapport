\chapter*{General Introduction} 
\label{chap:introduction} 
\addcontentsline{toc}{chapter}{General Introduction}

L'agriculture a largement contribué à l'économie de l'Algérie pendant des siècles. L'Algérie est l'un des principaux exportateurs de produits agricoles dans la région du Moyen-Orient et de l'Afrique du Nord. Le pays dispose d'une surface agricole totale de près de 7,2 millions d'hectares et est le 7e producteur de blé au monde (FAO, 2016).

Le secteur agricole algérien est principalement composé de petites exploitations familiales, qui représentent plus de 90\% de l'ensemble des exploitations du pays (FAO, 2018). Ce secteur est principalement axé sur les cultures céréalières telles que le blé, l'orge et l'avoine, ainsi que sur les cultures de tubercules telles que les pommes de terre, les oignons et l'ail. La production animale est également un sous-secteur important, les moutons, les chèvres, les volailles et les bovins étant les espèces les plus couramment élevées.

Ces dernières années, le gouvernement algérien a pris des mesures pour stimuler le secteur agricole en investissant dans la technologie, les infrastructures et la formation. Cela comprend l'établissement de nouveaux systèmes d'irrigation, l'utilisation de machines agricoles modernes et la promotion des meilleures pratiques agricoles. En outre, le gouvernement investit dans des initiatives de recherche et de développement pour augmenter les rendements et améliorer la sécurité alimentaire (Algeria Today, 2019).

Le gouvernement algérien a également mis en œuvre un certain nombre de politiques visant à encourager la production et les exportations agricoles. Il s'agit notamment du Plan national de développement agricole (PNDA), qui est conçu pour stimuler l'efficacité et la compétitivité (FAO, 2019). En outre, le gouvernement a introduit diverses subventions et autres incitations pour encourager les investissements dans le secteur.


Pour répondre à ces défis, l'intégration de technologies intelligentes dans l'agriculture, telles que les smart farms, peut être une solution prometteuse. Les smart farms utilisent des capteurs, des technologies de l'Internet des objets (IoT) et des algorithmes d'analyse de données pour surveiller et optimiser les processus agricoles, notamment l'irrigation, la fertilisation et la gestion des cultures. Cela permet aux agriculteurs de prendre des décisions éclairées en temps réel pour maximiser la productivité tout en réduisant les coûts et l'impact environnemental.



