\chapter*{Géneral Introduction} 
\label{chap:introduction} 
\addcontentsline{toc}{chapter}{Géneral Introduction}


\section*{Domaine d'etude}

L'agriculture a largement contribué à l'économie de l'Algérie pendant des siècles. L'Algérie est l'un des principaux exportateurs de produits agricoles.
Ces dernières années, le gouvernement algérien a pris des mesures pour stimuler le secteur agricole en investissant dans la technologie, les infrastructures et la formation et et il s'agit notamment du Plan national de développement agricole (PNDA), qui est conçu pour stimuler l'efficacité et la compétitivité. <<il a été décidé d’orienter les efforts vers une agriculture intelligente et solide face au changement climatique en tenant compte de l’environnement et de l’équilibre des écosystèmes sans négliger le gaspillage, grâce à une bonne gestion des excédents de production>>[1].


\section*{Problématique }
Les serres intelligentes(SI) et l'Internet des Objets (IoT) sont des technologies émergentes qui ont transformé l'agriculture. SI sont des structures utilisées pour cultiver des plantes sous un environnement contrôlé, offrant ainsi une solution efficace pour la production de cultures dans des conditions météorologiques défavorables ou dans des régions où les terres sont limitées.
\\
Les serres intelligentes utilisent des technologies de pointe pour optimiser les conditions de croissance des plantes. Ces technologies peuvent inclure des capteurs de température, d'humidité et de lumière, des systèmes d'irrigation automatisés et des logiciels d'analyse de données pour suivre les conditions de croissance des plantes.SI permettent également une surveillance et un contrôle précis des conditions environnementales.
\\
L'utilisation de l'IoT dans SI offre une connectivité en réseau aux appareils électroniques,des capteurs...,collectent une grande quantité de données,puis elles sont analysées et interprétées de manière efficace.
\\

\section*{Objectifs }
Pour répondre à ces défis, voici les objectifs suivants :
\\
-1 Optimiser la production : Les serres intelligentes peuvent aider les agriculteurs à optimiser leur production en régulant les conditions environnementales pour une croissance optimale des plantes. Les capteurs IoT peuvent être utilisés pour surveiller les niveaux d'humidité, de température et de lumière, ce qui permet aux agriculteurs de prendre des décisions éclairées sur l'irrigation, la ventilation et l'éclairage.
\\
-2 Réduire la consommation d'eau et d'énergie : SI peuvent également aider à réduire la consommation d'eau et d'énergie. Les capteurs IoT peuvent être utilisés pour surveiller les niveaux d'humidité du sol et de l'air, ce qui permet aux agriculteurs de réguler l'irrigation et la ventilation de manière efficace. De plus, SI peuvent être équipées de technologies d'éclairage LED et d'autres technologies d'efficacité énergétique pour réduire la consommation d'énergie.
\\
-3 System automatisés : le systeme s'interagit auto au temps reel . 
\section*{Organisation de mémoire }
Notre mémoire est structuré comme suite : 
\\
\textbf{Chapitre 1: }
Information generale sur les Serres 
\\
\textbf{Chapitre 2: }
IOT
\\
\textbf{Chapitre 3: }
Conception et Analyse
\\
\textbf{Chapitre 4: }
Implimentation