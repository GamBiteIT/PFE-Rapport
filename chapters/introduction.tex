\chapter*{General Introduction} 
\label{chap:introduction} 
\addcontentsline{toc}{chapter}{General Introduction}


\section*{Domaine d'etude}

L'agriculture a largement contribué à l'économie de l'Algérie pendant des siècles. L'Algérie est l'un des principaux exportateurs de produits agricoles dans la région du Moyen-Orient et de l'Afrique du Nord. Le pays dispose d'une surface agricole totale de près de 7,2 millions d'hectares  (FAO, 2016).
Ces dernières années, le gouvernement algérien a pris des mesures pour stimuler le secteur agricole en investissant dans la technologie, les infrastructures et la formation et et il s'agit notamment du Plan national de développement agricole (PNDA), qui est conçu pour stimuler l'efficacité et la compétitivité (FAO, 2019) .<<il a été décidé d’orienter les efforts vers une agriculture intelligente et solide face au changement climatique en tenant compte de l’environnement et de l’équilibre des écosystèmes sans négliger le gaspillage, grâce à une bonne gestion des excédents de production>>[1].


\section*{Problématique }
Comment utiliser les technologies pour améliorer la productivité et l'efficacité des exploitations agricoles en Algérie, tout en surmontant les défis liés à l'accès à l'eau, la gestion des sols et des ressources, la logistique et la formation des agriculteurs?

L'agriculture en Algérie est confrontée à plusieurs défis, notamment la rareté de l'eau, la dégradation des sols, la faible productivité, la gestion inefficace des ressources et la dépendance à l'égard des importations alimentaires. 

\section*{Objectifs }
Pour répondre à ces défis, l'intégration de technologies intelligentes dans l'agriculture, telles que les fermes intelligentes, peut être une solution prometteuse. Les fermes intelligentes utilisent des capteurs, des technologies de l'Internet des objets (IoT) et des algorithmes d'analyse de données pour surveiller et optimiser les processus agricoles, notamment l'irrigation, la fertilisation et la gestion des cultures. Cela permet aux agriculteurs de prendre des décisions éclairées en temps réel et un accès à distance pour surveiller et contrôler l’environnement cela pour maximiser la productivité tout en réduisant les coûts et l'impact environnemental.


\section*{Organisation de mémoire }
Notre mémoire est structuré comme suite : 
\\
\textbf{Chapitre 1: }
hhh
\\
\textbf{Chapitre 2: }
ggg
\\
\textbf{Chapitre 3: }
nn