\chapter{Présentation général de l’IOT} \label{chap:Présentation général de l’IOT}
\section{Définition de IOT}
L'Internet des objets (IdO) est un système d'appareils informatiques interconnectés, de machines mécaniques et numériques, d'objets, d'animaux ou de personnes avec des identifiants uniques (UID) qui nécessitent une communication inter-humaine. Interaction humaine Nécessite une connexion humaine. Interaction homme ou homme-ordinateur.[6] ses fondamentaux caractéristiques sont comme shivalibhadaniya écrit  [7]: \newline
1- Connectivité \newline
La connectivité est une exigence importante de l’infrastructure IoT. Les objets de l’IoT doivent être connectés à l’infrastructure IoT. N’importe qui, n’importe où, n’importe quand peut se connecter, cela devrait être garanti à tout moment. Par exemple, la connexion entre les personnes via des appareils Internet tels que les téléphones mobiles et d’autres gadgets, ainsi que la connexion entre les appareils Internet tels que les routeurs, les passerelles, les capteurs, etc.  \newline                            
2- Intelligence et identité \newline
L’extraction de connaissances à partir des données générées est très importante. Par exemple, un capteur génère des données, mais ces données ne seront utiles que si elles sont correctement interprétées. Chaque appareil IoT a une identité unique. Cette identification est utile pour suivre l’équipement et parfois pour interroger son état.\newline
 
3- Évolutivité \newline
Le nombre d’éléments connectés à la zone IoT augmente de jour en jour. Par conséquent, une configuration IoT devrait être capable de gérer l’expansion massive. Les données générées en tant que résultat sont énormes et doivent être traitées de manière appropriée.\newline
 
4- Dynamique et auto-adaptatif (complexité) \newline
 Les appareils IoT doivent s’adapter de manière dynamique aux contextes et scénarios changeants. Supposons une caméra destinée à la surveillance. Il doit être adaptable pour travailler dans différentes conditions et différentes situations d’éclairage (matin, après-midi, nuit).\newline
 
5- Architecture \newline
l’architecture IoT ne peut pas être de nature homogène. Il doit être hybride, prenant en charge les produits de différents fabricants pour fonctionner dans le réseau IoT. IoT n’appartient à aucune branche d’ingénierie. L’IoT est une réalité lorsque plusieurs domaines se rejoignent.\newline
 
6- Sécurité \newline
Il existe un risque que les données personnelles sensibles des utilisateurs soient compromises lorsque tous leurs appareils sont connectés à Internet. Cela peut entraîner une perte pour l’utilisateur. La sécurité des données est donc le défi majeur. De plus, l’équipement impliqué est énorme. Les réseaux IoT peuvent également être à risque. Par conséquent, la sécurité des équipements est également essentielle.